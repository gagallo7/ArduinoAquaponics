\begin{description}
    \item[Arduino UNO] \hfill \\
        Some project authors recommends the Arduino MEGA because of its extra GPIO pins.
        But we only have the UNO version by now.
    \item[DC Motor] \hfill \\
        A simple DC Motor can be enough for this project.
        It could be used to feed the fish periodically.

        There is a simple mechanism inspired by the video \cite{dcmotor},
        where the fish food is wrapped in a pot and rotated down just for a arbitrary short time,
        and then rotated back up.
        It can be controlled by sending electrical current timed by the Arduino.
    \item[Waterproof Temperature Sensor] \hfill \\
        This item is necessary for monitoring whether the fish's ambient is favorable for the fish.
    \item[Water Level Sensor] \hfill \\
    \item[Water Pump] \hfill \\
        Needed to give potential energy to the water flow,
        being fundamental to the water's cycle.
    \item[pH and ORP probe] \hfill \\
        pH levelling is an essential feature of the system.
        The fish, the nitro-bacterias and the plants needs to live in a specific pH-range ambient.
        With the probe,
        when the ambient is suffering with a pH decreasing,
        the system could automatically drop some amount of CaCO3 into the water to rise the pH from the fish tank,
        for example.
    \item[Relay Board] \hfill \\
        Some items,
        like the Water Pump,
        draws too much current if compared with Arduino's capacity.
        So one needs to use relays to connected another power source with the Arduino's output signals.
\end{description}
