% All important information for this work
% Like Arduino Architecture
\subsection{Background}

An extensive research has been made aiming at some basic Aquaponics knowledge and previous Aquaponics automation projects.
The idea for creating a period water cycle has came from \cite{simpleArduinoAquaponics},
where a simple project is proposed and an Arduino only automates this very feature of the system.
But the system design of the former project with 3 tanks and 1 second water cycle will not be applied,
the \cite{Kretzinger2015} uses a simpler approach with only 2 tanks and a periodic water cycle,
with the difference that the latter's period is much longer.

There are some projects that uses multiples microcontrollers,
like \cite{GarethColeman2014},
which uses an Arduino Mega with Raspberry PI.
For the sake of simplification,
this project focuses on using only one microcontroller.

\subsection{Technologies}
\subsubsection{Arduino}
Arduino is an open-source prototyping platform that provides general purpose input/output (GPIO) ports to be controlled by a programmable microprocessor \cite{arduino_intro}.
The GPIO feature allows the platform to control several output like: Light-Emitting Diodes (LEDs) brightness,
Motors velocity and even publish something online.
Some input control examples are: read a sensor value,
detect a touch on a touch-enabled screen, 
read an email and detect button interactions.

To program the microcontroller,
only a USB cable and a program uploader is needed,
the latter is provided by the Arduino IDE,
which is an open-source software.
This interface could be used for direct serial communication between Computer and the board.

There are shields which are hardware expansions to the default platform.
Shields can be connected to the main Arduino board to grant new features,
such as Ethernet communication.

This platform is massively used in various types of electronic projects ---
such as 3D printing, IoT, robotics and embedded systems ---
since it is an inexpensive hardware with a participative open-source community,
and lots of contributed tutorials and libraries available.
Besides,
the Arduino IDE runs in most computers,
because is compatible with Linux, MacOS and Windows.

The Arduino version used in this project is the Arduino UNO Revision 3.
The table \ref{uno_table} shows the main characteristics from this board.

\begin{table}[h]
\centering
\caption{Main Arduino UNO specifications \cite{arduino_overview}}
\label{uno_table}
\begin{tabular}{ll}
\textbf{Microcontroller}             & ATmega328P                                           \\
\textbf{Operating Voltage}           & 5V                                                   \\
\textbf{Input Voltage (recommended)} & 7-12V                                                \\ \textbf{Input Voltage (limit)}       & 6-20V                                                \\
\textbf{Digital I/O Pins}            & 14 (of which 6 provide PWM output)                   \\
\textbf{PWM Digital I/O Pins}        & 6                                                    \\
\textbf{Analog Input Pins}           & 6                                                    \\
\textbf{DC Current per I/O Pin}      & 20 mA                                                \\
\textbf{DC Current for 3.3V Pin}     & 50 mA                                                \\
\textbf{Flash Memory}                & 32 KB (ATmega328P)of which 0.5 KB used by bootloader \\
\textbf{Clock Speed}                 & 16 MHz                                              
\end{tabular}
\end{table}

\subsubsection{Sensors}
The pH probe is an equipment that can be submerged into the water to measure its pH level.
It is necessary for this project as a required component of \ref{req2}.

\begin{figure}[h]
    \centering
    \includegraphics[width=.5\textwidth]{img/phProbe.jpg}
    \caption{pH Probe at the top, protoboard interface at the bottom left and the USB interface at the bottom right \cite{dfrobot_analog}}
    \label{fig:phProbe}
\end{figure}

It is difficult to find the pH Sensor individually, 
the entire probe-kit is sold.
The pH probe shown in the figure \ref{fig:phProbe} is originated from the DFRobot factory.
To extract the actual pH level from the input voltage,
an example code was provided,
the reasoning follows: take 10 measurements,
sort them,
take the 6 values from the center to remove noisy values,
average the extract values and apply the formula \ref{pH_formula}.

\begin{align}
    millivolts &= \frac{averageVoltageValue \times 5}{1024 \times 6}  \label{pH_formula} \\
    pH &= 3.5 \times millivolts
\end{align}


\subsection{Related work}
There are some automated Aquaponics projects available on the Internet,
but most of them doesn't have a good documentation.
So it has been needed to grab parts of information among every material found on Internet.
One of the best sources found was from a hackaday's post from \cite{GarethColeman2014},
which describes with decent detail how they achieved the construction of their Arduino-powered Aquaponics system.

The solution for the requirement of periodic water cycle \ref{req1} is based on the cstewart000's Instructable \cite{simpleArduinoAquaponics},
where the objective is to make a very simple automated system,
with only \ref{req1} as case use.
Although this work shows the recommended time proportion between water cycling activation and deactivation of 15 minutes per hour,
as well as a simple code.

Additionally,
the water pH Control feature \ref{req2} is extracted from Simonson's solution \cite{elder_austin_simonson_hyduino},
which presents the idea of using peristaltic pumps to regulate the pH level.

In order to demonstrate the proof of concept without any physical building,
there is a video with a Graphical User Interface in the section \ref{sec:video},
which the user can simulate how the system responds to the undesired pH values in the tank.
This video is made based on the IP Capture library for Processing \cite{ipcapture_2016},
that makes feasible to render life video from a IP camera,
hence it is possible to watch what happens with the Arduino in real-time when one changes a sensor value.
