%Sometimes it is a good idea to put domain objects in \texttt{}
%The template and the descriptions are based on the book Applying UML and Patterns: 
%An Introduction to Object-Oriented Analysis and Design and Iterative Development
%(3rd Edition) by Craig Larman.
\begin{usecase}

    \addlongtitle{Keep water tank's}{pH level in 6-7 range}{\quad Details} 

%Scope: the system under design
%\addfield{Scope:}{System-wide}

%Level: "user-goal" or "subfunction"
%\addfield{Level:}{User-goal}

%Primary Actor: Calls on the system to deliver its services.
\additemizedfield{Actors:}
{
\item Microcontroller
\item Peristaltic Water Pumps 
\item pH Sensors
}

%Stakeholders and Interests: Who cares about this use case and what do they want?
%\additemizedfield{Stakeholders and Interests:}{
	%\item Stakeholder 1 name: his interests
	%\item Stakeholder 2 name: his interests
%}

%Preconditions: What must be true on start and worth telling the reader?
%\addfield{Preconditions:}{}
%when multiple
\additemizedfield{Preconditions:}{
    \item The Microcontroller, pH sensors and the peristaltic pumps must be installed in the system
    \item A proper relay must be installed in the system
         OR a power transistor to supply 12V DC
    \item The external power supply and the water pump must be compatible
} 

%Postconditions: What must be true on successful completion and worth telling the reader
\addfield{Postconditions:}
{
    The water tank's pH level is between 6 and 7.
}
%when multiple
%\additemizedfield{Preconditions:}{}

%Main Success Scenario: A typical, unconditional happy path scenario of success.
\addscenario{Main Success Scenario:}{
	\item The Microcontroller checks the pH sensor.
    \item The pH output is between 6 and 7.
	%\item The Microcontroller executes a run cycle of 25\% in the PWM in a 1 hour period.
}

%Extensions: Alternate scenarios of success or failure.
\addscenario{Extensions:}{
	\item[2.a] Lower pH values in output:
		\begin{enumerate}
            \item[1.] The circuit sends this signal to start the high pH relay.
            \item[2.] The relay distributes the needed current to the high pH's peristaltic pump.
            \item[3.] High pH water is blended with the water tank's, until the latter gets into the normal pH range.
            \item[4.] The circuit sends this signal to stop the high pH relay.
		%\item[2.] User returns to step 1
		\end{enumerate}
	\item[2.a] Higher pH values in output:
		\begin{enumerate}
            \item[1.] The circuit sends this signal to start the low pH relay.
            \item[2.] The relay distributes the needed current to the low pH's peristaltic pump.
            \item[3.] High pH water is blended with the water tank's, until the latter gets into the normal pH range.
            \item[4.] The circuit sends this signal to stop the low pH relay.
		%\item[2.] User returns to step 1
		\end{enumerate}
	%\item[5.a] Invalid subsriber data:
		%\begin{enumerate}
		%\item[1.] System shows failure message
		%\item[2.] User returns to step 2 and corrects the errors
		%\end{enumerate}
}

%Special Requirements: Related non-functional requirements.
\additemizedfield{Special Requirements:}{
	%\item R1: Operation Time Limit requirement
	\item \ref{req2}
}

% TODO: COOL Technology and Data Variations List: Varying I/O methods and data formats.
%\addscenario{Technology and Data Variations List:}{
	%\item[1a.] Alternative first action with other technology
%}

%Frequency of Occurrence: Influences investigation, testing and timing of implementation.
%\addfield{Frequency of Occurrence:}{}

%Miscellaneous: Such as open issues/questions
%\addfield{Open Issues:}{}

\end{usecase}
