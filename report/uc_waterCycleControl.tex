%Sometimes it is a good idea to put domain objects in \texttt{}
%The template and the descriptions are based on the book Applying UML and Patterns: 
%An Introduction to Object-Oriented Analysis and Design and Iterative Development
%(3rd Edition) by Craig Larman.
\begin{usecase}

\addtitle{Water cycle control}{Template test} 

%Scope: the system under design
\addfield{Scope:}{System-wide}

%Level: "user-goal" or "subfunction"
\addfield{Level:}{User-goal}

%Primary Actor: Calls on the system to deliver its services.
\addfield{Actors:}{Microcontroller}

%Stakeholders and Interests: Who cares about this use case and what do they want?
\additemizedfield{Stakeholders and Interests:}{
	\item Stakeholder 1 name: his interests
	\item Stakeholder 2 name: his interests
}

%Preconditions: What must be true on start and worth telling the reader?
%\addfield{Preconditions:}{}
%when multiple
\additemizedfield{Preconditions:}{
    \item The Microcontroller must be installed in the system
    \item Water pump and the PWM module are working normally
} 

%Postconditions: What must be true on successful completion and worth telling the reader
\addfield{Postconditions:}
{
    The fish tank's water should be pumped to the vegetable media each quarter of an hour.
}
%when multiple
%\additemizedfield{Preconditions:}{}

%Main Success Scenario: A typical, unconditional happy path scenario of success.
\addscenario{Main Success Scenario:}{
	\item The Microcontroller executes a run cycle of 25\% in the PWM in a 1 hour period.
	\item The circuit sends this signal to a relay.
    \item The relay activates the water pump at a 15 minutes per hour rate.
}

%Extensions: Alternate scenarios of success or failure.
\addscenario{Extensions:}{
	\item[2.a] Invalid login data:
		\begin{enumerate}
		\item[1.] System shows failure message
		\item[2.] User returns to step 1
		\end{enumerate}
	\item[5.a] Invalid subsriber data:
		\begin{enumerate}
		\item[1.] System shows failure message
		\item[2.] User returns to step 2 and corrects the errors
		\end{enumerate}
}

%Special Requirements: Related non-functional requirements.
\additemizedfield{Special Requirements:}{
	\item first applicable non-functional requirement
	\item second applicable non-functional requirement
}

%Technology and Data Variations List: Varying I/O methods and data formats.
\addscenario{Technology and Data Variations List:}{
	\item[1a.] Alternative first action with other technology
}

%Frequency of Occurrence: Influences investigation, testing and timing of implementation.
\addfield{Frequency of Occurrence:}{}

%Miscellaneous: Such as open issues/questions
%\addfield{Open Issues:}{}

\end{usecase}
