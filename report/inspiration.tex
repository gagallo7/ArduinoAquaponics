There are some automated aquaponics projects available on the Internet,
but most of them doesn't have a good documentation.
So it has been needed to grab parts of information among every material found on Internet.
One of the best sources found was from a hackaday's post from \cite{GarethColeman2014},
which describes with decent detail how they achieved the construction of their Arduino-powered aquaponics system.

Lots of research have been done about the theoretical aspect of the Aquaponics.
In \cite{GoddekDelaideMankasinghEtAl2015},
the authors show high complexity problems involving mechanisms to achieve pH equilibrium for optimizing the quality of life for the fish,
plants and nitro-bacterias,
since each living component of the system lives well in a certain pH-Range.
So there is a challenge to separate the pH level by region.
Despite of the requirement \ref{req2} treats the pH level of all mediums monolithically,
the dedicated controlling of each medium's pH level is not treated in this work,
for the sake of simplicity.

\subsection{Reasons to build an Aquaponics system}

A great amount of the published projects has a commercial goal:
to make an efficient and small-sized system that can afford to produce organic products in a large scale.

On other hand,
in the \cite{GoddekDelaideMankasinghEtAl2015} there is a try to address the sustainability aspect of the Aquaponics.
This aspect stands for making a low and efficient nutrient input into the system and making a minimal environment footprint.

\subsection{Differences between Hidroponics and Aquaponics}

The Hidroponics is a system that uses a nutritive water to feed the plants.
It is a inorganic system, 
where the addition of inorganic nutrients is needed and the main live component is the plant.
On the other hand, the Aquaponics is a partly-organic system cite,
where the fish is added to the system,
and its waste,
the ammonia,
serves as a nutrient to the plants.

The great advantage of the Hidroponics over the Aquaponics is that the last may have some issues with human diseases,
like the presence of snails with parasites in the fish tank or some water-borne disease.

\subsection{Guidelines}
A Do It Yourself (DIY) online magazine called Make,
has an article  that presents some rules to make a durable aquaponics.
