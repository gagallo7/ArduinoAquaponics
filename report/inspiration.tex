There are some automated aquaponics projects available on the Internet,
but most of them doesn't have a reasonable good documentation.
So it has been needed to grab parts of information among every material found on Internet.
One of the best sources found was from a hackaday's post from \cite{gareth_coleman_aquapionics_2016},
which describes with decent detail how they achieved the construction of their Arduino-powered aquaponics system.

Lots of research have been done about aquaponics.
In \cite{goddek2015challenges}, the authors show high complexity problems involving mechanisms to achieve pH equilibrium for optimizing the quality of life for the fish, plants and nitro-bacterias.
Since each living component of the system lives well in a certain pH-Range.
So there is a challenge to separate the pH level by region.

\subsection{Why people are interested in Aquaponics?}

A great amount of the published projects has a commercial goal:
to make an efficient and small-sized system that can afford to produce organic products in a large scale.

On other hand,
in the \cite{goddek2015challenges} there is a try to address the sustainability aspect of the aquaponics
This aspect stands for making a low and efficient nutrient input into the system and making a minimal environment footprint.
