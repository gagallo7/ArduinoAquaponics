\subsection{Motivation}
Basically,
the automation purpose is to save labor and add more security,
precision and velocity to tasks that formerly were made by humans or other animals.

If someone wonder about the reason for automating tasks,
this person needs to pay attention about living in modern civilizations.
When one walks on the sidewalk,
one can see vehicles parked on the road,
these are automated objects,
because saves the labor of walking to distant places,
via the electronic management of several components,
such as acceleration,
fuel measurement,
injection,
breaking with ABS technology etc.
But not only cars are automated.
The microwave,
airplanes and elevators are automated objects which are used frequently.

Generally, 
an automation project is composed by a microcontroller, 
electronic devices and a variable number of sensors.
The former is the main component,
which is responsible to manage the system by reading the input from the latter,
and then process what to do,
so it can send signals to electronic devices to accomplish the system's goal.
This processing is made with programming,
a programmer can install the code inside the microcontroller,
which will execute the program when turned on.
So it is presupposed that the programmer will align the program with the system's intention.

\subsubsection{Context}
When someone builds an Aquaponics system,
one needs to manage it everyday,
since there is a plenty of work to do for maintaining the system work,
because it deals with living beings.
So it is a time wasting management,
with lots of repetitive tasks,
which one can be understood as periodic events.

Some of these tasks can be reliably automated,
together with others that involves problems that may occur during the system's growth.
For example, the fish feeding is a periodic event,
but the water's pH control is not a periodic event,
instead it is classified as an adverse event.

Basically this work is trying to decrease human intervention in an Aquaponics system via automatic management of adverse and periodic events,
in order to reduce the time wasted to take care of the system, 
as well as to make the management more reliable.


\subsection{Challenges}

\subsubsection{Without automation}
To build an Aquaponics system,
one needs to be precise and cautious when one makes the design of the project,
there are lots of requirements that need to be respected in order to make the Aquaponics work.
The most important requirements are strongly connected with equilibrium,
because we are dealing with a biological system.
For example,
the proportion between the fish tank and the plant's grow bed tank are vital to optimize.
If one gets the wrong proportion,
the system won't work,
both the fish and the plants will die precociously \cite{Leatherbury2014}.

Another problem is to distribute the byproducts of each medium to another one,
the main byproduct of the fish tank is ammonia,
if the concentration of the latter increases,
the fish will die by high toxicity,
hence this ammonia has to be delivered to the bacteria,
who will digest this substance via the nitrogen cycle,
producing nitrate.

As the final product are comestibles,
there is another concern: food toxicity.
One cannot use components made with materials that can cause water contamination,
because the fish and the plant will be contaminated as well.
So there is the need to use non-toxic elements in the system.
Besides there is also a high concern with the water source,
the water cannot be infected with pathogens that causes diseases,
like amoeba.

\subsubsection{With automation}
In order to automate an Aquaponics system,
one must take care with microelectronic components,
they have to be protected from water and humidity.

Second,
there is the need to use relays to add high current-drain devices,
like the submersible water pump,
which is essential to the system.
If this advise is ignored,
the microcontroller will be damaged.
