\subsection{Motivation}
Basically,
the automation purpose is to save labor and add more security,
precision and velocity to tasks that formerly were made by humans or other animals.

If someone wonder about the reason for automating tasks,
this person needs to pay attention about living in modern civilizations.
When one walks on the sidewalk,
one can see vehicles parked on the road,
these are automated objects,
because saves the labor of walking to distant places --
or making animals to walk for us --
via the electronic management of several components,
such as acceleration,
fuel measurement,
injection,
breaking with ABS technology etc.
But not only cars are automated.
The microwave,
airplanes and elevators are automated objects which are used frequently.

Generally, 
an automation project is composed by a microcontroller, 
electronic devices and a variable number of sensors.
The former is the main component,
which is responsible to manage the system by reading the input from the latter,
and then process what to do,
so it can send signals to electronic devices to accomplish the system's goal.
This processing is made with programming.
A programmer can install the code inside the microcontroller,
which will execute the program when turned on.
So it is presupposed that the programmer will align the program with the system's intention.

\subsubsection{Context}
When someone builds an Aquaponics system,
one needs to manage it everyday,
since there is a plenty of work to do for maintaining the system work,
due the dependency of living beings on the system.
So it is a time wasting management,
with lots of repetitive tasks,
which one can be understood as periodic events.
Other tasks needs precision,
such as water leveling and pH balancing,
that feature can be achieved with automation.

Some of these tasks can be reliably automated,
together with others that involves problems that may occur during the system's growth.
For example, the fish feeding is a periodic event,
but the water's pH control is not a periodic event,
instead it is classified as an adverse event.

Basically this work is trying to decrease human intervention in an Aquaponics system via automatic management of adverse and periodic events,
in order to reduce the time wasted to take care of the system, 
as well as to make the management more reliable.

The figure \ref{fig:aquaponicsExample} demonstrate an example of a functional Indoor Growing System applying Aquaponics.

\paragraph{Reasons to build an Aquaponics system}

A great amount of the published projects has a commercial goal:
to make an efficient and small-sized system that can afford to produce organic products in a large scale.

On other hand,
in the \cite{GoddekDelaideMankasinghEtAl2015} there is an effort to address the sustainability aspect of the Aquaponics.
This aspect stands for making a low and efficient nutrient input into the system and making a minimal environment footprint.

\paragraph{Differences between Hidroponics and Aquaponics}

The Hidroponics is a system that uses a nutritive water to feed the plants.
It is a inorganic system, 
where the addition of inorganic nutrients is needed and the main live component is the plant.
On the other hand, the Aquaponics is a partly-organic system cite,
where the fish is added to the system,
and its waste,
the ammonia,
serves as a nutrient to the plants.

The great advantage of the Hidroponics over the Aquaponics is that the last may have some issues with human diseases,
like the presence of snails with parasites in the fish tank or some water-borne disease. \cite{wilson2005greenhouse}.

\begin{figure}[h]
    \centering
    \includegraphics[width=.5\textwidth]{img/aquaponics.png}
    \caption{An example of a built Aquaponics system. The plant medium is above and at the right of the electronic pump. At the bottom is the fish tank. \cite{goldstein2013indoor}}
    \label{fig:aquaponicsExample}
\end{figure}


\subsection{Challenges}

\subsubsection{Without automation}
To build an Aquaponics system,
one needs to be precise and cautious when one makes the design of the project,
since there are lots of requirements that need to be respected in order to make the Aquaponics work.
The most important requirements are strongly connected with equilibrium,
because we are dealing with a biological system.
For example,
the proportion between the fish tank and the plant's grow bed tank are vital to optimize.
If one gets the wrong proportion,
the system won't work,
both the fish and the plants will die precociously \cite{Leatherbury2014}.

Another problem is to distribute the byproducts of each medium to another one,
the main byproduct of the fish tank is ammonia,
if the concentration of the latter increases,
the fish will die by high toxicity,
hence this ammonia has to be delivered to the bacteria,
who will digest this substance via the nitrogen cycle,
producing nitrate.

As the final product are comestibles,
there is another concern: food toxicity.
One cannot use components made with materials that can cause water contamination,
because the fish and the plant will be contaminated as well.
So there is the need to use non-toxic elements in the system.
Besides there is also a high concern with the water source,
the water cannot be infected with pathogens that causes diseases,
like amoeba.

Lots of research have been done about the biological aspect of the Aquaponics.
In \cite{GoddekDelaideMankasinghEtAl2015},
the authors show high complexity problems involving mechanisms to achieve pH equilibrium for optimizing the quality of life for the fish,
plants and nitro-bacteria,
since each living component of the system lives well in a certain pH-Range.
So there is a challenge to separate the pH level by region.
Despite of the requirement \ref{req2} treats the pH level of all mediums monolithically,
the dedicated controlling of each medium's pH level is not treated in this work,
for the sake of simplicity.

\subsubsection{With automation}
In order to automate an Aquaponics system,
one must take care with microelectronic components,
they have to be protected from water and humidity.

Second,
there is the need to use relays to add high current-drain devices,
like the submersible water pump,
which is essential to the system.
If this advise is ignored,
the microcontroller will be damaged.
